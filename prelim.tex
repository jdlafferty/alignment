%!TEX root=./main.tex
\section{Preliminaries}

We consider a two-layer neural network $f:\R^d \to \R$ with the hidden-layer width $p$
\begin{align}\label{eqn:nonlinear-network}
    f(x) = \frac{1}{\sqrt p}\sum_{r=1}^p\beta_r\psi(w_r\transpose x)= \frac{1}{\sqrt p}\beta\transpose\psi(Wx),
\end{align}
where $\psi$ denotes the element-wise activation function, $W = (w_1,...,w_p)\transpose\in\Rpd$ and $\beta = (\beta_1,...,\beta_p)\transpose\in\Rp$ are the feed-forward weights for first and second layers.
Given data $\{(x_1,y_1),\ldots,(x_n,y_n)\}$, we hope to minimize the L2 loss
\begin{equation}
    \Loss(W,\beta) = \frac{1}{2}\sum_{i=1}^n \big(f(x_i)-y_i\big)^2
\end{equation}
The gradient descent algorithm for minimizing the loss writes
\begin{align*}
    \beta_r(t+1) &= \beta_r(t)-\eta\frac{\partial\Loss}{\partial \beta_r}(W(t),\beta(t)), \\
    w_r(t+1) &= w_r(t)-\eta\frac{\partial\Loss}{\partial w_r}(W(t),\beta(t)), \quad r=1,\ldots,p.
\end{align*}
where $\eta>0$ is the step size, and we initialize $\beta(0)$ and $w_r(0)$ with standard Gaussian vectors. We also introduce the notations $u(t),e(t)\in \R^n$, where $u_i(t)$ is the network output on input $x_i$ at step $t$, $e_i(t) = u_i(t)-y$ is corresponding prediction error.

\paragraph{Backpropagation on two-layer network.}

Backpropagation has been the most commonly used scheme for computing the gradient of the network parameters. In particular, for two-layer network \eqref{eqn:nonlinear-network}, the gradients given by backpropagation are
\begin{equation*}
    \frac{\partial\Loss}{\partial \beta_r} = \frac{1}{\sqrt p}\sum_{i=1}^n e_i\psi(w_r\transpose x_i), \quad
    \frac{\partial\Loss}{\partial w_r} = \frac{1}{\sqrt p} \sum_{i=1}^n e_i \beta_r\psi'(w_r\transpose x_i)x_i.
\end{equation*}
However, such learning algorithm requires updating its weights with backpropagated error signals that consist of non-local information of weight parameters from top layers. More specifically, given an input $x_i$, the gradient of second-layer weights $\frac{\partial \Loss}{\partial \beta_r}$ contains only the local inputs $\psi(w_r x_i)$ and error signal $e_i = f(x_i) - y_i$, while the gradient of first-layer weights $\frac{\partial \Loss}{\partial w_r}$ requires not only the local input $x_i$, local output $w_r x_i$ but also the backpropagated error signal $e_i\beta_r$, which contains the information on the second-layer forward weights $\beta_r$. This appearance of $\beta_r$ is due to the chain rule between two adjacent layers and can be recognized as the backward weights in reverse propagation of error signals.
Consequently, backpropagation requires the forward weights between layers to be identical with the backward weights. However, from a biological perspective, this is a requirement on synaptic symmetry and is regarded as not biological plausible \citep{lillicrap2016random}. 

\paragraph{Feedback alignment on two-layer network.}

One of the alternative approaches proposed to improve the biological plausibility over backpropagation is \emph{Feedback Alignment}, which at the backpropagation stage replaces the forward weights at each layer by an independent fixed random backpropagation weights \citep{lillicrap2016random}. In particular, for the two-layer networks considered in this work, the second-layer weights $\beta$ is replaced by a fixed random backpropagation weights $b\in\Rp$ at backpropagation stage while the gradient update on $\beta$ remains the same, \ie,
\begin{align}\label{eqn:alignment-update}
    \delta_{FA}(w_r) = \frac{1}{\sqrt{p}} \sum_{i=1}^n e_i b_r\psi'(w_r\transpose x_i)x_i.
\end{align}
After the substitution, the feedback alignment update on $W$ no longer depends on $\beta$ directly and therefore bears no requirement on synaptic symmetry. Note that the path of $\beta(t)$ does not follow the one by gradient descent under standard backpropagation when considered jointly with the feedback alignment updates on $W(t)$. It is important to notice that the feedback alignment algorithm that we end up arriving at does not explicitly optimize on any particular loss function even though we motivate the algorithm from a loss function $\Loss$ and the corresponding gradient descent algorithm.
\cref{algo:fa} shows the feedback alignment algorithm on \eqref{eqn:nonlinear-network}.

%\begin{minipage}{\textwidth}
\begin{algorithm}[H]
\centering
\caption{Feedback Alignment on Tow-layer Networks}\label{algo:fa}
    \begin{algorithmic}[1]
        \Require activation $\psi$, dataset $\{(x_i,y_i)\}_{i=1}^n$, and step size $\eta$.
        \State {\bf initialize} $W(0)$, $\beta(0)$ and $b$ with standard Gaussian entries
        \While{not converge}
            \State $\beta_r(t+1) \gets \beta_r(t) - \frac{\eta}{\sqrt p} \sum_{i=1}^n e_i(t)\psi(w_r(t)\transpose x_i)$,
            \State $w_r(t+1) = w_r(t) - \frac{\eta}{\sqrt{p}} \sum_{i=1}^n e_i(t) b_r\psi'(w_r(t)\transpose x_i)x_i$, for $r=1,2,\ldots,p$.
        \EndWhile
    \end{algorithmic}    
\end{algorithm}%
%\end{minipage}


