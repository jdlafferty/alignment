% Macros
\usepackage{amsfonts}
\usepackage{amsmath}
\usepackage{amsthm}
\usepackage{authblk}
\usepackage{amssymb}
\usepackage{mathrsfs}
\usepackage{bm}
\usepackage{bbm}
\usepackage{mathtools}
\usepackage{fancyhdr}
\usepackage{algorithm}
\usepackage{algpseudocode}
\usepackage[frak=esstix]{mathalfa}
\usepackage{xspace}
\usepackage{xfrac}
\usepackage{afterpage}
\usepackage{framed}
\usepackage[inline]{enumitem}
\usepackage[colorlinks]{hyperref}
	\hypersetup{linkcolor=[RGB]{200,40,41}}
	\hypersetup{citecolor=[RGB]{66,113,174}}
	\hypersetup{urlcolor=[RGB]{231,197,71}}
\usepackage[acronym,smallcaps,nowarn]{glossaries}
\usepackage{lmodern}

\usepackage{natbib}
\usepackage{upgreek}
\usepackage{booktabs}

\usepackage[svgnames]{xcolor}
    \definecolor{notecolor}{RGB}{137,89,168}
    \definecolor{quotecolor}{RGB}{66,113,174}
    \definecolor{warningcolor}{RGB}{249,145,87}
    \definecolor{YaleBlue}{RGB}{0, 53, 107}
    \definecolor{YaleBlueLight}{RGB}{40,109,192}
    \definecolor{YaleBlueLighter}{RGB}{99,170,255}
    \definecolor{YaleGreen}{RGB}{95, 113, 45}
    \definecolor{YaleOrange}{RGB}{189, 83, 25}
\usepackage[all]{hypcap}
%\usepackage[color]{showkeys}
%    \definecolor{refkey}{RGB}{137,89,168}
%    \definecolor{labelkey}{RGB}{137,89,168}
\usepackage{seqsplit}
%\usepackage{showkeys}
\usepackage{xstring}

%\renewcommand*\showkeyslabelformat[1]{%
%\noexpandarg%
%% instead of \textvisiblespace you can also put in ~
%% if you want to keep a plain space at space characters
%\StrSubstitute{\(\)#1\(\)}{ }{\textvisiblespace}[\TEMP]%
%\fbox{\parbox[t]{.80\marginparwidth}{\raggedright\normalfont\small\ttfamily\expandafter\seqsplit\%expandafter{\TEMP}}}}

\usepackage{graphicx}

\usepackage[capitalize]{cleveref}
\crefname{ineqn}{Ineq.}{Ineqs.}
\creflabelformat{ineqn}{(#2{\upshape#1}#3)}

%\makeatletter
%  \SK@def\Cref#1{\SK@\SK@@ref{#1}\SK@Cref{#1}}%
%\makeatother
%
%\makeatletter
%\SK@def\cref#1{\SK@\SK@@ref{#1}\SK@cref{#1}}%
%\makeatother

\newcommand{\cf}{\emph{cf.}\!\xspace}
\newcommand{\ie}{\emph{i.e.}\xspace}
\newcommand{\eg}{\emph{e.g.}\xspace}
\newcommand{\iid}{i.i.d.\!\xspace}

\newcommand{\mmin}{\mathrm{min}}
\newcommand{\mmax}{\mathrm{max}}
\newcommand\transpose{^{\mathpalette\raiseT{\scriptstyle\intercal}}}
\newcommand\raiseT[2]{\raisebox{0.2ex}{$#1#2$}}
\DeclarePairedDelimiterX{\innerDelim}[2]{\langle}{\rangle}{#1, #2}
\newcommand{\inner}[3][]{\innerDelim[#1]{#2}{#3}}

\newcommand{\reals}{{\mathbb{R}}}
\newcommand{\complex}{{\mathbb{C}}}
\newcommand{\Rm}{{\mathbb{R}^m}}

\DeclareMathOperator{\range}{\mathrm{range}}

\newcommand{\eps}{\varepsilon}
\newcommand{\lambdamax}{\lambda_\mathrm{max}}
\newcommand{\lambdamin}{\lambda_\mathrm{min}}

\newcommand{\ith}{$i$-th }
\newcommand{\jth}{$j$-th }
\newcommand{\kth}{$k$-th }
\newcommand{\rth}{$r$-th }
\newcommand{\ellth}{$\ell$-th}
\newcommand{\ii}{\mathrm{i}}

\newcommand{\suchthat}{\mathrm{s.t.}}
\DeclareMathOperator*{\st}{\,:\,}
\DeclareMathOperator*{\given}{\,|\,}

\DeclarePairedDelimiter{\abs}{\lvert}{\rvert}
\DeclarePairedDelimiter{\norm}{\lVert}{\rVert}
\DeclareMathOperator*{\argmin}{arg\,min}
\DeclareMathOperator*{\argmax}{arg\,max}

\newcommand{\RR}{\mathbb{R}}
\newcommand{\Rnn}{{\mathbb{R}^{n\times n}}}
\newcommand{\Rnp}{{\mathbb{R}^{n\times p}}}
\newcommand{\Rdp}{{\mathbb{R}^{d\times p}}}
\newcommand{\Rpd}{{\mathbb{R}^{p\times d}}}
\newcommand{\Rmn}{{\mathbb{R}^{m\times n}}}
\newcommand{\Rnm}{\mathbb{R}^{n\times m}}
\newcommand{\Rnr}{\mathbb{R}^{n\times r}}
\newcommand{\Rrr}{\mathbb{R}^{r\times r}}
\newcommand{\Rpp}{\mathbb{R}^{p\times p}}
\newcommand{\Rmm}{{\mathbb{R}^{m\times m}}}
\newcommand{\Rmr}{\mathbb{R}^{m\times r}}
\newcommand{\Rrn}{{\mathbb{R}^{r\times n}}}
\newcommand{\Rd}{{\mathbb{R}^{d}}}
\newcommand{\Rp}{{\mathbb{R}^{p}}}
\newcommand{\Rk}{{\mathbb{R}^{k}}}
\newcommand{\Rdd}{{\mathbb{R}^{d\times d}}}
\newcommand{\RNN}{{\mathbb{R}^{N\times N}}}
\newcommand{\Rn}{{\mathbb{R}^n}}
\newcommand{\Rnd}{{\mathbb{R}^{n\times d}}}

\DeclareMathOperator{\expect}{\mathbb{E}}
\DeclareMathOperator{\prob}{\mathbb{P}}

\makeatletter
\@tfor\next:=ABCDEFGHIJKLMNOPQRSTUVWXYZ\do{%
  \def\command@factory#1{%
    \expandafter\def\csname cal#1\endcsname{{\mathcal{#1}}}
  }
 \expandafter\command@factory\next
}
\makeatother

\makeatletter
\@tfor\next:=ABCDEFGHIJKLMNOPQRSTUVWXYZ\do{%
  \def\command@factory#1{%
    \expandafter\def\csname #1#1\endcsname{{\mathbb{#1}}}
  }
 \expandafter\command@factory\next
}
\makeatother

\def\icomment#1{{\color{notecolor} [\textrm{#1}]}}
\def\wcomment#1{{\color{warningcolor} [#1]}}

\newcommand{\expp}[1]{{\exp\!\big({#1}\big)}}

\theoremstyle{plain}
\newtheorem{theorem}{Theorem}[section]
\newtheorem{lemma}[theorem]{Lemma}
\newtheorem{corollary}[theorem]{Corollary}
\newtheorem{conjecture}[theorem]{Conjecture}
\newtheorem{distribution}[theorem]{Distribution}
\newtheorem{proposition}[theorem]{Proposition}
\newtheorem{sdp}[theorem]{SDP}

\theoremstyle{definition}
\newtheorem{definition}[theorem]{Definition}
\newtheorem{example}[theorem]{Example}
\newtheorem{open}[theorem]{Open Problem}
\newtheorem{assumption}[theorem]{Assumption}

\theoremstyle{remark}
\newtheorem{remark}[theorem]{Remark}
\newtheorem{observation}[theorem]{Observation}

\newcommand{\defeq}{\coloneqq}
\newcommand{\eqdef}{\eqqcolon}
\newcommand\numberthis{\addtocounter{equation}{1}\tag{\theequation}}

\numberwithin{equation}{section}

\newcommand{\Var}{\mathbb{V}\mathrm{ar}}
\newcommand{\dif}{\,d}
\newcommand{\dmid}{\,\|\,}
\newcommand*\diff{\mathop{}\! d}

\newcommand{\T}{{\mathpalette\raiseT{\scriptstyle\intercal}}}
\newcommand{\R}{\mathbb{R}}
\newcommand{\E}{\mathbb{E}}
\newcommand{\I}{\mathbb{I}}
\newcommand{\thetat}{\tilde{\theta}}
\newcommand{\thetab}{\bar{\theta}}
\newcommand{\tlam}{\tilde{\lambda}}
\newcommand\Loss{\mathcal{L}}

\renewcommand{\algorithmicrequire}{\textbf{Input:}}
\renewcommand{\algorithmicensure}{\textbf{Output:}}

\def\chris#1{{\color{YaleBlueLight} [\textrm{Chris: #1}]}}
\def\ganlin#1{{\color{YaleGreen} [\textrm{Ganlin: #1}]}}
\def\john#1{{\color{YaleOrange} [\textrm{John: #1}]}}
