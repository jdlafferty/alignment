\begin{abstract}
Stochastic gradient descent with backpropagation is the workhorse of artificial neural networks. It has long been recognized that backpropagation fails to be a biologically plausible algorithm. Fundamentally, it is a non-local procedure---updating the weight between a presynaptic and postsynaptic neuron requires knowledge of the weights between the postsynaptic neuron and other neurons. This limits the use of artificial neural networks as a tool for understanding the biological principles of information processing in the brain. Lillicrap et al.~(2016) propose a more biologically plausible ``feedback alignment'' algorithm that uses random and fixed backpropagation weights, and show promising simulations. In this paper we study the mathematical properties of the feedback alignment procedure by analyzing convergence and alignment for two-layer networks under squared error loss. In the overparameterized setting, we prove that the error converges to zero exponentially fast, and also that regularization is necessary in order for the  parameters to become aligned with the random backpropagation weights. Simulations are given that are consistent with this analysis and suggest further generalizations. These results contribute to our understanding of how biologically plausible algorithms might carry out weight learning in a manner different from Hebbian learning, with performance that is comparable with the full non-local backpropagation algorithm.
\end{abstract}
