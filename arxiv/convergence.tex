%!TEX root=./main.tex

\section{Convergence with Random Backpropagation Weights}

Due to the replacement of backward weights with the random backpropagation weights, there is no guarantee \emph{a priori} that the algorithm will reduce the squared error loss $\Loss$. \citet{lillicrap2020backpropagation} study the convergence on two-layer linear networks in a continuous time setting. Through the analysis of a system of differential equations on the network parameters, convergence to the true linear target function is shown, in the population setting of  arbitrarily large training data.
Among recent studies of over-parametrized networks under backpropagation, the neural tangent kernel (NTK) is heavily utilized to describe the evolution of the network during training \citep{jacot2018neural}. For any neural network $f(x,\theta)$ with parameter $\theta$, the NTK is defined as
\begin{equation*}
	K_f(x,y) = \Big\langle \frac{\partial f (x,\theta)}{\partial \theta},\frac{\partial f (y,\theta)}{\partial \theta}\Big\rangle.
\end{equation*}
Given a dataset $\{(x_i,y_i)\}_{i=1}^n$, we can also consider its corresponding Gram matrix $K = (K_f(x_i,x_j))_{n\times n}$. \citet{jacot2018neural} show that in the infinite width limit, $K_f$ converges to a constant at initialization and does not drift away from initialization throughout training. In the over-parameterized setting, if the Gram matrix $K$ is positive definite, then $K$ will remain close to its initialization during training, resulting in linear convergence of the squared error loss \citep{du2018gradient,du2019gradient,gao2020model}.
% However, after replacing the feed-forward weights $\beta$ with a random backpropagation weights $b$, feedback alignment algorithm no longer follows a strict gradient descent on the squared error loss and the previous argument cannot be applied directly to our case.
For the two-layer network $f(x, \theta)$ defined in \eqref{eqn:nonlinear-network} with $\theta = (\beta,W)$, the kernel $K_f$ can be written in two parts, $G_f$ and $H_f$, which correspond to $\beta$ and $W$ respectively:
\begin{equation*}
K_f(x,y) = G_f(x, y) + H_f(x,y) \defeq \Big\langle \frac{\partial f (x,\theta)}{\partial \beta},\frac{\partial f (y,\theta)}{\partial \beta}\Big\rangle + \sum_{r=1}^p\Big\langle \frac{\partial f (x,\theta)}{\partial w_r},\frac{\partial f (y,\theta)}{\partial w_r}\Big\rangle.
\end{equation*}
Under the feedback alignment scheme with random backward weights $b$, $G_f$ remains the same as for standard backpropagation, while one of the gradient terms $\frac{\partial f}{\partial w_r}$ in $H_f$ changes to
$\widetilde{\frac{\partial f(x, \theta)}{\partial w_r}}   = \frac{1}{\sqrt{p}} b_r \psi'(w_r\transpose x)x,
$
with $H_f$ replaced by $H_f = \sum_{r=1}^p\Big\langle \widetilde{\frac{\partial f (x,\theta)}{\partial w_r}},\frac{\partial f (y,\theta)}{\partial w_r}\Big\rangle$.
As a result, $H_f$ is no longer positive semi-definite and close to $0$ at initialization if the network is over-parameterized. However, if $G = (G_f(x_i,x_j))_{n\times n}$ is positive definite and $H = (H_f(x_i,x_j))_{n\times n}$ remains small during training, we are still able to show that the loss $\Loss$ will converge to zero exponentially fast.

\begin{assumption}\label{assump:G}
Define the matrix $\overline{G} \in \R^{n\times n}$ with entries
$\overline{G}_{i,j} = \E_{w\sim \calN(0,I_p)}\psi(w\transpose x_i) \psi(w\transpose  x_j)$.
Then we assume that the minimum eigenvalue satisfies $\lambda_{\min}(\overline{G}) \geq \gamma$, where $\gamma$ is a positive constant.
\end{assumption}
\vskip5pt

\begin{theorem}\label{thm:nonliner_conv}
Let $W(0)$, $\beta(0)$ and $b$ have \iid standard Gaussian entries. Assume \textnormal{(1)} Assumption \ref{assump:G} holds, \textnormal{(2)} $\psi$ is smooth, $\psi$, $\psi'$ and $\psi''$ are bounded and \textnormal{(3)} $|y_i|$ and $\|x_i\|$ are bounded for all $i\in[n]$. Then there exists positive constants $c_1$, $c_2$, $C_1$ and $C_2$, such that for any $\delta\in(0,1)$, if $p \geq \max\left(C_1\frac{n^2}{\delta\gamma^2}, C_2\frac{n^4\log p}{\gamma^4}\right)$, then with probability at least $1-\delta$ we have that
\begin{equation}\label{eq:conv}
    \|e(t+1)\| \leq (1-\frac{\eta\gamma}{4})\|e(t)\|
\end{equation}
and
\begin{equation}
\label{eq:weights}
    \|w_r(t)-w_r(0)\| \leq c_1\frac{n\sqrt{\log p}}{\gamma\sqrt p}, \quad |\beta_r(t)-\beta_r(0)| \leq c_2\frac{n}{\gamma\sqrt p}
\end{equation}
for all $r\in[p]$ and $t>0$.
\end{theorem}

We note that the matrix $\overline{G}$ in Assumption \ref{assump:G} is the expectation of $G$ with respect to the random initialization, and is thus close to $\overline{G}$ due to concentration. To justify the assumption, we provide the following proposition.

\begin{proposition}\label{prop:positive-definiteness}
Suppose $x_1,...,x_n \overset{\iid}{\sim} \calN(0,I_d/d)$ and the activation function $\psi$ is sigmoid or tanh. If $d=\Omega(n)$, then Assumption \ref{assump:G} holds with high probability.
\end{proposition}

\noindent
The proofs of Theorem \ref{thm:nonliner_conv} and Proposition \ref{prop:positive-definiteness} are deferred to \cref{sec:appendix-convergence}.
